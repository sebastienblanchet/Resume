%%%%%%%%%%%%%%%%%%%%%%%%%%%%%%%%%%%%%%%%%%%%%%%%%%%%%%%%%%%%%%%%%%%%%%%%%%%%%%%%
% Medium Length Graduate Curriculum Vitae
% LaTeX Template
% Version 1.2 (3/28/15)
%
% This template has been downloaded from:
% http://www.LaTeXTemplates.com
%
% Original author:
% Rensselaer Polytechnic Institute
% (http://www.rpi.edu/dept/arc/training/latex/resumes/)
%
% Modified by:
% Sebastien Blanchet <s3blanch@edu.uwaterloo>
%
% Important note:
% This template requires the res.cls file to be in the same directory as the
% .tex file. The res.cls file provides the resume style used for structuring the
% document.
%
%  VERSION FRANÇAISE
%%%%%%%%%%%%%%%%%%%%%%%%%%%%%%%%%%%%%%%%%%%%%%%%%%%%%%%%%%%%%%%%%%%%%%%%%%%%%%%%

\documentclass[mm]{res}

% Default font is the helvetica postscript font
%\usepackage{helvet}
\usepackage{xspace}
\usepackage{fontawesome}
\usepackage[hidelinks]{hyperref}
\usepackage{pdfcomment}
\usepackage{epstopdf}
\usepackage{textcomp}
% fix Package inputenc Error: Invalid UTF-8 byte 147
\UseRawInputEncoding
%\usepackage{acronym}
\hypersetup{
  colorlinks=false,
}
\usepackage{geometry}
\geometry{
  letterpaper,
  left=0.5in,
  top=0.5in
}
\usepackage{graphicx}
\usepackage{xcolor}
\usepackage{etoolbox}
\usepackage[tooltip]{acro}

% Increase text height
\textheight=720pt
\textwidth=446pt

% Spacing in between jobs and projects
\parskip=6pt

%toggle
\newtoggle{color}
\toggletrue{color}
% \togglefalse{color}

\newtoggle{fr}
\toggletrue{fr}

% Shared custom commands
\input{../common/commands}

\begin{document}

% Shared header
\input{../common/header}

%---------------------------------------------------------------------------------------------------------

\begin{resume}

%---------------------------------------------------------------------------------------------------------
%   SKILLS
%---------------------------------------------------------------------------------------------------------
\npspctoprule
\section{\headingskills}
\tb \textbf{Programmation:} C embarqu\'e, C++, Python, MATLAB, LabVIEW\\
\tb \textbf{Hardware:} MCU (ARM, TI, Arduino, x86 Assembler, Raspberry Pi), FPGA (VHDL, Xilinx Vivado)\\
\tb \textbf{Simulation:} LabVIEW-FPGA/RT, Simulink, OPAL-RT, Speedgoat, dSPACE, SOLIDWORKS, ANSYS\\
\tb \textbf{OS:} Windows, macOS, Linux (Ubuntu, Red Hat), RTOS (FreeRTOS, Phar Lap ETS)\\
\tb \textbf{Protocoles:} CAN, LIN, UDS, SPI, I2C, JTAG, UART, USB, RS422, Ethernet, FTP, PCIe\\
\tb \textbf{Int\'egration:} Git Bash, Atlassian (JIRA, Confluence, Stash) GitHub, Jenkins, SCons, Maven\\
\tb \textbf{Concepts:} syst\`emes de contr\^ole discr\`ets, DSP, HIL/SIL, TDD, SCRUM, OOP, DSA, CI, API\\
\tb \textbf{Autres:} Bash, Vim, HTML5, CSS, JSON, XML, \LaTeX, Markdown, MISRA\xspace

%---------------------------------------------------------------------------------------------------------

\toprule

%---------------------------------------------------------------------------------------------------------
%   EDUCATION
%---------------------------------------------------------------------------------------------------------
\section{\headingeducation}
% Modify layout
\begin{format}
\employer{l}\location{r}\\
\title{l}\dates{r}\\
\end{format}  

\employer{\textbf{\href{https://uwaterloo.ca}{University of Waterloo}{\logouw}}}
\location{\textbf{Waterloo, ON, CAN}}
\title{Candidat pour B.ASc. \textsl{GPA: 3.5/4.0}\\
G\'enie M\'ecanique/M\'ecatronique, Co-op}
% janv.   f\'evr.   mars    avril   mai     juin    juil.   ao\^ut    sept.   oct.    nov.    d\'ec.
\dates{\textsl{sept. 2013 - avril 2019}}
\begin{position}
\end{position}

%--------------------------------------------------------------------------------------------------------- 

\toprule

%---------------------------------------------------------------------------------------------------------
%   EXPERIENCE
%---------------------------------------------------------------------------------------------------------
% \faSuitcase \faBlackTie
\section{\headingexperience}
% Modify layout
\begin{format}
\employer{l}\location{r}\\
\title{l}\dates{r}\\
\body\\
\end{format}

% Tesla: Co-op 9
\employer{\textbf{\href{https://www.tesla.com/}{Tesla}{\logotesla}}}
\location{\textbf{Palo Alto, CA, USA}}
\title{\textsl{G\'enie de Logiciels Embarqu\'e - Produits \'Energie}}
\dates{\textsl{sept. - d\'ec. 2018}}
\begin{position}
\tb Codage de micrologiciels en C embarqu\'e pour le contr\^ole d\textquotesingle \'electronique de puissance sur les DSP et MCU\\
\tb Exposition au paquet entier: RTOS, pilotes de ports s\'eriaux (UDS, CAN, SPI), application et diagnostiques\\
\tb D\'eployer un cadre self-test embarqu\'e C sur plusieurs ECUs pour eliminer les efforts manuel au chantier\\
\tb Am\'eliorer les outils de generation de code en Java et les test r\'egression avec Python Pytest\\
\tb Assurer l'int\'egration avec les outils Atlassian, Git Bash, revue de code, Jenkins, test unitaires, SIL, HIL
\end{position}

% Apple: Co-op 6 - 8 
\employer{\textbf{\href{https://www.apple.com}{Apple}{\logoapple}}}
\location{\textbf{Cupertino, CA, USA}}
\title{\textsl{G\'enie de Contr\^ole - Groupe de Projects Sp\'eciaux}}
\dates{\textsl{ao\^ut 2017 - ao\^ut 2018}}
\begin{position}
\tb D\'evelopper un syst\`eme HIL pour valider les algorithmes pour le contr\^ole d\textquotesingle \'electronique de puissance en C\\
\tb Emuler et optimiser les mod\`eles d\textquotesingle haute fid\'elit\'e sur FPGA Xilinx pour le contr\^ole de faible latence en $\mu$s\\
\tb D\'eployer un HMI LabVIEW pour la communication d\'eterministe entre PC, controlleur RTOS et FPGA\\
\tb Flasher le microcontr\^oleur des PCB par JTAG, ports seriaux et Ethernet avec la version r\'ecente de logiciel \\
\tb Appliquer la th\'eorie DSP pour convertir des mod\`eles et filtres Simulink au domaine discr\`et en C embarqu\'e\\
\tb Mis en oeuvre un cadre de testage Python automatis\'e pour l'int\'egration continuel et r\'egression du logiciel
\end{position}

% Altaeros: Co-op 5
\employer{\textbf{\href{http://www.altaeros.com}{Altaeros}{\logoaltaeros}}}
\location{\textbf{Boston, MA, USA}}
\title{\textsl{G\'enie de Syst\`emes - Recherche et D\'eveloppement}}
\dates{\textsl{janv. - avril 2017}}
\begin{position}
\tb Effectuer des analyses num\'eriques en Python pour le syst\`eme \'electrom\'ecanique d'un a\'erostat\\
\tb Utiliser l'\'equipment de laboratoire \'electronique et un HMI LabVIEW pour enregistrer des donn\'ees de test
\end{position}

% ODI: Co-op 4
\employer{\textbf{\href{https://www.ontariodie.com}{Ontario Die International}{\logoodi}}}
\title{\textsl{Design M\'ecanique - Recherche et D\'eveloppement}}
\dates{\textsl{mai - ao\^ut 2016}}
\begin{position}
% \tb Designed robotic components (electrical, hydraulic) of PLC/CNC bending systems in SOLIDWORKS
\end{position}

% PWC: Co-op 3
\employer{\textbf{\href{http://www.pwc.ca}{Pratt \& Whitney Canada}{\logopwc}}}
\location{\textbf{Mississauga, ON, CAN}}
\title{\textsl{Gestion de Programme - Op\'erations Turbosoufflantes}}
\dates{\textsl{sept. - d\'ec. 2015}}
\begin{position}
% \tb Assured on time OEM delivery of a quality turbofan engine while meeting their expectations and needs
\end{position}

% Skyjack: Co-op 2
\employer{\textbf{\href{http://www.skyjack.com}{Linamar}{\logolinamar}}}
\location{\textbf{Guelph, ON, CAN}}
\title{\textsl{G\'enie de Fabrication - Skyjack}}
\dates{\textsl{janv. - avril 2015}}
\begin{position}
% \tb Worked with a team of engineers to troubleshoot production issues at an aerial work platform manufacturer
\end{position}

%---------------------------------------------------------------------------------------------------------

\toprule

%---------------------------------------------------------------------------------------------------------
%   PROJECTS
%---------------------------------------------------------------------------------------------------------
\section{\headingprojects}
% Modify layout
\begin{format}
\title{l}\employer{l}\dates{r}\\
\body\\
\end{format}

% % TEMPLATE [X]
% \title{\textsl{Latest Proj}}
% \employer{\textsl{X}}
% \dates{\textsl{Ongoing}}
% \begin{position}
% \tb next
% \end{position}
% Portfolio [SW]

% Robot Multi [CTL]
\title{\textsl{Robot Arm Controller}}
\employer{\textsl{ECE 488: Multi-Variable Controls}}
\dates{\textsl{Ongoing}}
\begin{position}
\tb Modeling MIMO non-linear system in MATLAB, designing controller using advanced state-space methods
\end{position}

% FYDP [CTL]
\title{\textsl{Heated Press System}}
\employer{\textsl{ME 482: Capstone Design Project}}
\dates{\textsl{Ongoing}}
\begin{position}
\tb Leading electrical system efforts: harnessing, temperature and motor controls with Arduino
\end{position}


% DSA [SW]
\title{\textsl{MIT Open Courseware Self-Study}}
\employer{\textsl{6.006 Introduction to Algorithms}}
\dates{\textsl{mai 2018}}
\begin{position}
\tb Covered complexity, sorting algorithms, graphs, and dynamic programming in Jupyter Python notebooks
\end{position}

% Ball and Beam [CTL]
\title{\textsl{Ball \& Beam Lab}}
\employer{\textsl{ECE481: Digital Control Systems}}
\dates{\textsl{ao\^ut 2017}}
\begin{position}
\tb Designed LabVIEW HMI, performed system ID, implemented/tuned digital controller on NI cRIO FPGA
\end{position}

% Drumming Hack [CTL]
\title{\textsl{Drum Rhythm Arduino Hack}}
\employer{\textsl{Personal: WIT Hackathon}}
\dates{\textsl{Mar. 2017}}
\begin{position}
\tb Coded embarqu\'e in C and communicated over UART to MATLAB for real-time monitoring of vibration
\end{position}

% Wind turbine [CTL]
\title{\textsl{Wind Turbine Pitch Actuator}}
\employer{\textsl{ME360: Control Systems}}
\dates{\textsl{d\'ec. 2016}}
\begin{position}
\tb Studied time/frequency domain responses in MATLAB for closed-loop stability of PI controlled Simulink
\end{position}


% DC Motor [CTL]
\title{\textsl{DC Motor Control System}}
\employer{\textsl{ME360: Control Systems}}
\dates{\textsl{oct. 2016}}
\begin{position}
\tb Designed PID control in Simulink for a DC motor; implemented in real-time with QUARC C code generation
\end{position}


% Dune Buggy [HO]
\title{\textsl{Dune-Buggy Magneto Repair}}
\employer{\textsl{Personal}}
\dates{\textsl{ao\^ut 2016}}
\begin{position}
\tb Diagnosed fuel system ignition issue then replaced coil and armature of solid-state system
\end{position}

%---------------------------------------------------------------------------------------------------------

\toprule

%---------------------------------------------------------------------------------------------------------
%   INTERESTS
%---------------------------------------------------------------------------------------------------------
\section{\headinginterests}
\tb Further developing skills related to embarqu\'e, electronics, machine learning and embedded systems\\
􏰚\tb R\'eparer des v\'ehicules hors-terrain, projets \'electr\^oniques, hockey, golf, natation et socialisation (bilingue)
%---------------------------------------------------------------------------------------------------------

\end{resume}
\end{document}
