%%%%%%%%%%%%%%%%%%%%%%%%%%%%%%%%%%%%%%%%%%%%%%%%%%%%%%%%%%%%%%%%%%%%%%%%%%%%%%%%
% Medium Length Graduate Curriculum Vitae
% LaTeX Template
% Version 1.2 (3/28/15)
%
% This template has been downloaded from:
% http://www.LaTeXTemplates.com
%
% Original author:
% Rensselaer Polytechnic Institute
% (http://www.rpi.edu/dept/arc/training/latex/resumes/)
%
% Modified by:
% Sebastien Blanchet <s3blanch@edu.uwaterloo>
%
% Important note:
% This template requires the res.cls file to be in the same directory as the
% .tex file. The res.cls file provides the resume style used for structuring the
% document.
%
%  VERSION FRANÇAISE
%%%%%%%%%%%%%%%%%%%%%%%%%%%%%%%%%%%%%%%%%%%%%%%%%%%%%%%%%%%%%%%%%%%%%%%%%%%%%%%%

\documentclass[mm]{res}

% Default font is the helvetica postscript font
%\usepackage{helvet}
\usepackage{xspace}
\usepackage{fontawesome}
\usepackage[hidelinks]{hyperref}
\usepackage{pdfcomment}
\usepackage{epstopdf}
\usepackage{textcomp}
% fix Package inputenc Error: Invalid UTF-8 byte 147
\UseRawInputEncoding
%\usepackage{acronym}
\hypersetup{
  colorlinks=false,
}
\usepackage{geometry}
\geometry{
  letterpaper,
  left=0.5in,
  top=0.5in
}
\usepackage{graphicx}
\usepackage{xcolor}
\usepackage{etoolbox}
\usepackage[tooltip]{acro}

% Increase text height
\textheight=720pt
\textwidth=446pt

% Spacing in between jobs and projects
\parskip=5pt

%toggle
\newtoggle{color}
\toggletrue{color}
% \togglefalse{color}

\newtoggle{fr}
\toggletrue{fr}

% Shared custom commands
\input{../common/commands}

\begin{document}

% Shared header
\input{../common/header}

%---------------------------------------------------------------------------------------------------------

\begin{resume}

%---------------------------------------------------------------------------------------------------------
%   SKILLS
%---------------------------------------------------------------------------------------------------------
\npspctoprule
\section{\headingskills}
\tb \textbf{Programmation:} C, C++, C\#, JavaScript, TypeScript, Python, MATLAB, HTML5/CSS3, Ruby, GraphQL\\
\tb \textbf{Frameworks:} Angular, Vue.js/Vuetify, Node.js, .NET/.NET Core, Rails, Jest\\
\tb \textbf{Libraries:} React, Boostrap, jQuery, MySQL, MongoDB, Express, boost, GNU, OpenCV\\
\tb \textbf{Hardware:} MCU (ARM, Texas Instrument, Arduino, Raspberry Pi), FPGA (VHDL, Xilinx Vivado)\\
\tb \textbf{Simulation:} LabVIEW-FPGA/RT, Simulink, OPAL-RT, Speedgoat, dSPACE, SOLIDWORKS, ANSYS\\
\tb \textbf{OS:} Windows, macOS, Linux (CentOS, RHEL), RTOS (FreeRTOS, Phar Lap ETS, TI-RTOS, QNX), UNIX\\
% \tb \textbf{Protocoles:} HTTP, TCP/IP, UDP, ARINC, CAN, LIN, UDS, SPI, I2C, JTAG, UART, USB, RS422, FTP\\
\tb \textbf{Concepts:} contr\^ole discr\`ets embarqu\'e, PID, DSP, HIL/SIL, TDD, OOP, DSA, CI/CD, REST API\\
\tb \textbf{Autres:} Git, Travis-CI, Jenkins, Docker, Atlassian, Bash, Vim, JSON, XML, \LaTeX, Markdown
%---------------------------------------------------------------------------------------------------------

\toprule

%---------------------------------------------------------------------------------------------------------
%   EDUCATION
%---------------------------------------------------------------------------------------------------------
\section{\headingeducation}
% Modify layout
\begin{format}
\employer{l}\location{r}\\
\title{l}\dates{r}\\
\end{format}

\employer{\textbf{\href{https://uwaterloo.ca}{University of Waterloo}{\logouw}}}
\location{\textbf{Waterloo, ON, CAN}}
\title{Baccalaur\'eat en Sciences Appliqu\'ees avec Distinction\\
G\'enie M\'ecanique/M\'ecatronique, Co-op \textsl{GPA: 3.5/4.0}}
\dates{\textsl{sept. 2013 - avril 2019}}
\begin{position}
\end{position}

%--------------------------------------------------------------------------------------------------------- 

\toprule

%---------------------------------------------------------------------------------------------------------
%   EXPERIENCE
%---------------------------------------------------------------------------------------------------------
% \faSuitcase \faBlackTie
\section{\headingexperience}
% Modify layout
\begin{format}
\employer{l}\location{r}\\
\title{l}\dates{r}\\
\body\\
\end{format}

% ing\'enierie
% Ing\'enierie

% FTE
\employer{\textbf{\href{http://www.pwc.ca}{Pratt \&  Whitney Canada}{\logopwc}}}
\location{\textbf{Montr\'eal, QC, CAN}}
\title{\textsl{Ing\'enieur Logiciel - Support aux Banc D'essais}}
\dates{\textsl{ao\^ut 2019 - pr\'esent}}
\begin{position}
\tb D\'eveloppement de logiciels pour EEC avec technologies modernes (Node.js, Vue.js, TypeScript, GraphQL)\\
\tb Contribution aux applications back-end multi-threaded en C ++/C\# pour CentOS/RHEL Linux\\
\tb Utilisation de programmation sockets TCP/UDP pour communiquer avec bases de donn\'ees MySQL/MongoDB\\
\tb Int\'egration d'un domaine Agile \`a l'aide de Git, Jenkins et divers cadres de tests unitaire/e2e\\
\tb Conception de pilotes en C embarqu\'e sur RTOS Linux pour l'acquisition/traitement de donn\'ees EEC
\end{position}

% Tesla: Co-op 9
\employer{\textbf{\href{https://www.tesla.com/}{Tesla}{\logotesla}}}
\location{\textbf{Palo Alto, CA, \'E-U}}
\title{\textsl{Ing\'enierie Logiciel Embarqu\'e (Co-op) - \'Energie}}
\dates{\textsl{sept. - d\'ec. 2018}}
\begin{position}
\tb Coder des micrologiciels en C embarqu\'e pour le contr\^ole d\textquotesingle \'electronique de puissance sur les DSP et MCU\\
\tb Exposition au paquet entier: RTOS, API pilotes de ports s\'eriel (UDS, CAN, SPI), application et diagnostiques\\
\tb D\'eployer un cadre self-test embarqu\'e C sur plusieurs ECU pour \'eliminer les efforts manuels au chantier\\
% \tb Am\'eliorer les outils de g\'en\'eration de code en Java et les test r\'egression avec Python Pytest\\
\tb Employer le ``test-driven development" en \'ecrivant des test unitaires, simulations SIL/HIL et r\'egression\\
\tb Assurer l'int\'egration dans un milieu Agile avec les outils Atlassian, Git Bash, revue de code/PR et Jenkins
\end{position}

% Apple: Co-op 6 - 8 
\employer{\textbf{\href{https://www.apple.com}{Apple}{\logoapple}}}
\location{\textbf{Cupertino, CA, \'E-U}}
\title{\textsl{Ing\'enierie Contr\^ole (Co-op) - Projets Sp\'eciaux}}
\dates{\textsl{ao\^ut 2017 - ao\^ut 2018}}
\begin{position}
\tb D\'evelopper un syst\`eme HIL pour valider les algorithmes pour le contr\^ole d\textquotesingle \'electronique de puissance en C\\
\tb \'Emuler et optimiser les mod\`eles d'haute fid\'elit\'e sur FPGA Xilinx pour le contr\^ole de faible latence en $\mu$s\\
\tb D\'eployer un HMI LabVIEW pour la communication d\'eterministe entre PC, contr\^oleur RTOS et FPGA\\
\tb Flasher le microcontr\^oleur des PCB par JTAG, ports s\'eriel et Ethernet avec la version r\'ecente de logiciel \\
\tb Appliquer la th\'eorie DSP pour convertir des mod\`eles et filtres Simulink au domaine discr\`et en C embarqu\'e
% \tb R\'ealiser un cadre de testage Python automatis\'e pour l'int\'egration continuel et la r\'egression du logiciel
\end{position}

% Altaeros: Co-op 5
\employer{\textbf{\href{http://www.altaeros.com}{Altaeros}{\logoaltaeros}}}
\location{\textbf{Boston, MA, \'E-U}}
\title{\textsl{Ing\'enierie Syst\`eme (Co-op) - R \& D}}
\dates{\textsl{janv. - avril 2017}}
\begin{position}
\tb Effectuer des analyses num\'eriques en Python pour le syst\`eme \'electrom\'ecanique d'un a\'erostat
% \tb Utiliser l'\'equipment de laboratoire \'electronique et un HMI LabVIEW pour enregistrer des donn\'ees de test
\end{position}

% ODI: Co-op 4
\employer{\textbf{\href{https://www.ontariodie.com}{Ontario Die International}{\logoodi}}}
\location{\textbf{Waterloo, ON, CAN}}
\title{\textsl{Ing\'enierie M\'ecanique (Co-op) - R \& D}}
\dates{\textsl{mai - ao\^ut 2016}}
\begin{position}
\tb Con\c{c}u des composants robotiques (\'electrique, hydraulique) de syst\`emes PLC/CNC avec SOLIDWORKS
\end{position}

% PWC: Co-op 3
\employer{\textbf{\href{http://www.pwc.ca}{Pratt \&  Whitney Canada}{\logopwc}}}
\location{\textbf{Mississauga, ON, CAN}}
\title{\textsl{Gestion de Programme (Co-op) - Op\'erations}}
\dates{\textsl{sept. - d\'ec. 2015}}
\begin{position}
\tb Assurer la livraison en temps des turbosoufflantes en d\'epassant les attentes et besoins de l'OEM en fran\c{c}ais
% en fonction de leurs exigences
\end{position}

% Skyjack: Co-op 2
\employer{\textbf{\href{http://www.skyjack.com}{Linamar}{\logolinamar}}}
\location{\textbf{Guelph, ON, CAN}}
\title{\textsl{Ing\'enierie de Fabrication (Co-op) - Skyjack}}
\dates{\textsl{janv. - avril 2015}}
\begin{position}
\tb Travailler avec une \'equipe d\textquotesingle ing\'enieurs pour d\'epanner des probl\`emes sur la ligne de fabrication
\end{position}

%---------------------------------------------------------------------------------------------------------

\toprule

%---------------------------------------------------------------------------------------------------------
%   PROJECTS
%---------------------------------------------------------------------------------------------------------
\section{\headingprojects}
% Modify layout
\begin{format}
\title{l}\employer{l}\dates{r}\\
\body\\
\end{format}

% % TEMPLATE [X]
% \title{\textsl{Latest Proj}}
% \employer{\textsl{X}}
% \dates{\textsl{En Cours}}
% \begin{position}
% \tb next
% \end{position}
% Portfolio [SW]

% Portfolio [SW]
\title{\textsl{Application Web Portfolio}}
\employer{\textsl{Personel}}
\dates{\textsl{En cours}}
\begin{position}
\tb Pr\'esentation de travaux personel avec une application Vue.js deploy\'e par Travis-CI avec GitHub-pages
\end{position}

% TODO point to new portfolio page
\title{\textsl{Cours Divers}}
\employer{\textsl{LinkedIn Learning}}
\dates{\textsl{En cours}}
\begin{position}
\tb Couvrant: Vue.js, Node.js, TypeScript, React, MySQL, Ruby, GraphQL (\textsl{\href{\myport}{voir portfolio}})
\end{position}

% Robot Multi [CTL]
\title{\textsl{Contr\^oleur de Bras Robotique}}
\employer{\textsl{ECE 488: Contr\^ole Multi-Variable}}
\dates{\textsl{avril 2019}}
\begin{position}
\tb Mod\'elisation et contr\^ole d'un syst\`eme MIMO non lin\'eaire avec MATLAB en utilisant des m\'ethodes avanc\'ees
\end{position}

% FYDP [CTL]
\title{\textsl{Syst\`eme de Press Chauff\'ee}}
\employer{\textsl{ME 482: Projet Capstone}}
\dates{\textsl{mars 2019}}
\begin{position}
\tb Chef des \'efforts du syst\`eme \'electrique: contr\^ole de la temp\'erature et du moteur
\end{position}

% Portfolio [SW]
% \title{\textsl{Entra\^inement d'Application Swift}}
% \employer{\textsl{Apple: Software University}}
% \dates{\textsl{ao\^ut 2018}}
% \begin{position}
% \tb Revue des fondamentaux du ``object oriented programming" en Swift et codage d'un application
% \end{position}

% DSA [SW]
% \title{\textsl{MIT ``Open Courseware"}}
% \employer{\textsl{6.006 Introduction aux Algorithmes}}
% \dates{\textsl{mai 2018}}
% \begin{position}
% \tb Revue la complexit\'e, algorithmes de tri, graphiques et programmation dynamique avec Python
% \end{position}
% Dune Buggy [HO]

\title{\textsl{R\'eparation Dune-Buggy}}
\employer{\textsl{Personnel}}
\dates{\textsl{ao\^ut 2018}}
\begin{position}
\tb Diagnostiquer le syst\`eme d'allumage de carburant et remplacer la bobine et l'armature
\end{position}

% Ball and Beam [CTL]
\title{\textsl{Laboratoire Ballon et Poutre}}
\employer{\textsl{ECE481: Contr\^ole Discr\`et}}
\dates{\textsl{ao\^ut 2017}}
\begin{position}
\tb Design d'un HMI LabVIEW, effectu\'e l'identification, r\'ealiser un contr\^oleur digitale sur NI cRIO FPGA
\end{position}

% Drumming Hack [CTL]
% \title{\textsl{Moniteur d'un Tambour}}
% \employer{\textsl{Personnel: WIT Hackathon}}
% \dates{\textsl{mars 2017}}
% \begin{position}
% \tb Coder le micrologiciels en C embarqu\'e et communiqu\'e les donn\'ees par UART avec MATLAB en temps r\'eel
% \end{position}

% % Wind turbine [CTL]
% \title{\textsl{Actionneur d\textquotesingle \'Eolienne}}
% \employer{\textsl{ME360: Contr\^ole Analogue}}
% \dates{\textsl{d\'ec. 2016}}
% \begin{position}
% \tb \'Etudier les r\'eponses du domaine temps/fr\'equence avec MATLAB pour assurer la stabilit\'e du contr\^oleur PI
% \end{position}

% DC Motor [CTL]
% \title{\textsl{Contr\^ole d'un Moteur DC}}
% \employer{\textsl{ME360: Contr\^ole Analogue}}
% \dates{\textsl{oct. 2016}}
% \begin{position}
% \tb Design d'un contr\^oleur PID en Simulunk; r\'ealiser en temps r\'eel avec la g\'en\'eration de code C QUARC
% \end{position}

%---------------------------------------------------------------------------------------------------------

\toprule

%---------------------------------------------------------------------------------------------------------
%   INTERESTS
%---------------------------------------------------------------------------------------------------------
\section{\headinginterests}
% \tb D\'evelopper les comp\'etences li\'e aux syst\`emes embarqu\'es, \'electroniques, apprentissage automatique \\
\tb R\'eparer des v\'ehicules hors-terrain, projets \'electr\^oniques, hockey, golf, natation et socialisation (bilingue)
%---------------------------------------------------------------------------------------------------------

\end{resume}
\end{document}
